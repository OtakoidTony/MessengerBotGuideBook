%!TEX encoding = UTF-8 Unicode
\documentclass[10pt,a4paper,final]{book}
\usepackage[scale=0.75,twoside,bindingoffset=5mm]{geometry}
\usepackage{amsmath}
\usepackage{amsfonts}
\usepackage{amssymb}
\usepackage{graphicx}
\usepackage{kotex}


\author{Rojiku, OtakoidTony, Park Hyun}
\title{Messenger Bot Tutorial}
\usepackage{minted}
\usepackage[onehalfspacing]{setspace}
\setstretch{1.5}

\usepackage{listings}

\usepackage{color}
\definecolor{lightgray}{rgb}{.9,.9,.9}
\definecolor{darkgray}{rgb}{.4,.4,.4}
\definecolor{purple}{rgb}{0.65, 0.12, 0.82}

\lstdefinelanguage{JavaScript}{
  keywords = {
      typeof, const, new, true,
      false, catch, function, return,
      null, catch, switch, var, if, in,
      while, do, else, case, break
  },
  keywordstyle=\color{blue}\bfseries,
  ndkeywords={class, export, boolean, throw, implements, import, this},
  ndkeywordstyle=\color{darkgray}\bfseries,
  identifierstyle=\color{black},
  sensitive=false,
  comment=[l]{//},
  morecomment=[s]{/*}{*/},
  commentstyle=\color{purple}\ttfamily,
  stringstyle=\color{black}\ttfamily,
  morestring=[b]',
  morestring=[b]"
}

\lstset{
   language=JavaScript,
   backgroundcolor=\color{lightgray},
   extendedchars=true,
   basicstyle=\footnotesize\ttfamily\setstretch{1},
   showstringspaces=false,
   showspaces=false,
   numbers=left,
   numberstyle=\footnotesize,
   numbersep=9pt,
   tabsize=2,
   breaklines=true,
   showtabs=false,
   captionpos=b
}

\begin{document}
\maketitle
Tutorials based on AutoReplyBot using javascript.
\chapter{}
\section{Setup}
챗봇을 작성하기에 앞서 구글 플레이스토어에서 아래 나열한 어플리케이션을 모두 설치를 하여라.
만약에 휴대폰 운영체제가 Android계열이 아닌 경우에는 컴퓨터를 이용해 Android을 실행하여 
챗봇을 구현할 수 있다. 컴퓨터가 만일 Odroid를 사용중이라면 스마트폰처럼 사용하여도 문제는
없다.

\setstretch{1}
\begin{description}
\item[MessengerBot]\hfill\\
com.xfl.kakaotalkbot
\item[Wear OS by Google]\hfill\\
com.google.android.wearable.app
\end{description}
\setstretch{1.5}
두 어플리케이션의 관계에 대해 설명하자면 MessengerBot에서 
Wear OS by Google을 이용해 메신저에 접근하는 관계라 할 수 있으리라.
독자도 알다시피 Wear OS by Google라는 어플리케이션은 스마트폰의 알림을
Wearable Device에 전송하는 기능과 스마트폰 대신 Wearable Device만을
이용해 메신저에 답장을 보내는 기능이 탑재되어 있다.
이를 이용하여 MessengerBot에서 Wear OS by Google를 이용해 자동응답을
할 수 있도록 구현한 것이다. 두 어플리케이션에 관한 설명은 차후 이어서 하도록 하고,
만약에 두 어플리케이션 모두 설치를 하였다면, 손 아이콘을 눌러 알림권한설정에서 앞에서 설치한
MessengerBot과 Wear OS by Google를 활성화를 하여라. 이걸로 준비 작업을 마친다.

\section{Make your script}
준비 작업을 마친 후, 우측 하단의 십자버튼을 누르고 독자가 원하는 스크립트 이름을 정하여라.
저자의 경우 'normal'이라고 정하였다.
아래 소스는 스크립트를 처음 작성할 때 자동으로 생성되는 스크립트이며,
본 책에서는 여러 이유로 주석을 제거한 상태이다.

\begin{lstlisting}[language=JavaScript,escapeinside=~~,caption={normal.js}]
const scriptName="normal.js";

function response(room, msg, sender, isGroupChat, replier, ImageDB, packageName, threadId){

}

function onStartCompile(){
    
}

function onCreate(savedInstanceState,activity) {
    var layout=new android.widget.LinearLayout(activity);
    layout.setOrientation(android.widget.LinearLayout.HORIZONTAL);
    var txt=new android.widget.TextView(activity);
    txt.setText("~\textcolor{black}{액티비티 사용 예시입니다.}~");
    layout.addView(txt);
    activity.setContentView(layout);
}
function onResume(activity) {}
function onPause(activity) {}
function onStop(activity) {}
\end{lstlisting}

우리는 개발을 그렇게 깊게 하지는 않을 것이기 때문에 몇 줄을 삭제하고자 한다.
내가 챗봇을 개발할 때는 자동으로 작성된 함수 중에서는 오직 response()함수만을 이용했다.

\begin{lstlisting}[language=JavaScript,escapeinside=~~]
function response(room, msg, sender, isGroupChat, replier, ImageDB, packageName, threadId){
    
}
\end{lstlisting}

이제 한결 나아진 것 같다. 이제, Hello World! 예제를 작성해보자.

\section{Hello World!}
\begin{lstlisting}[language=JavaScript,escapeinside=~~]
function response(room, msg, sender, isGroupChat, replier, ImageDB, packageName, threadId){
    replier.reply("Hello World!")
}
\end{lstlisting}
위와 같이 작성한 다음, 컴파일을 하여라.
컴파일은 저장버튼을 길게 누르면 할 수 있으며, 다른 방법으로는
스크립트가 나열된 메인화면에서 새로고침버튼을 누르면 컴파일할 수 있다.
컴파일을 하였으면, 메인화면으로 나와 벌레버튼을 눌러 Debug Room으로 진입하여라.
우선은, Debug Room에서 정상적으로 작동하는지를 살펴볼 것이다.
Debug Room에 진입하였으면 아무 말이나 입력하고 반응을 살펴보자.
그러면, 독자가 무엇을 입력하든간에 "Hello World!"라고만 응답하는 것을 볼 수 있을 것이다.
아직까지는 기능이 없는데다가 누가 무슨 말을 하든간에 "Hello World!"라고 메시지를 전송하는
문제가 다분한 챗봇이다. 따라서 이 봇이 특정 키워드에만 반응을 하도록 수정을 해야한다. 
어떻게 하면 특정 키워드에만 챗봇이 작동할 수 있을까?
이에 대해서 다음 장에서 설명하고자 한다.
\chapter{}
\section{Specifical Replication}
우리는 앞에서 챗봇이 무슨 말을 하든간에 "Hello World!"라고 메시지를 전송하는
문제를 마주하였다. 하지만, 이는 if문이라는 것을 이용하여 쉽게 해결할 수 있다.
if문은 매우 많이 사용되기에 반드시 알아야하는 구문 중 하나이다.
다음은 if문을 이용한 매우 간단한 예시이다.
\begin{lstlisting}[language=JavaScript,escapeinside=~~]
if (true) {
    print("Hello World!");
}
\end{lstlisting}
이것 뿐이다. 조건이 참(true)이기 때문에 당연히 Hello World!를 출력할 것이다.
이를 이용하여 우리는 위에서의 문제를 해결할 수 있다.
다음 따라오는 스크립트를 따라해보자.
\begin{lstlisting}[language=JavaScript,escapeinside=~~]
function response(room, msg, sender, isGroupChat, replier, ImageDB, packageName, threadId){
    if (msg=='Hello'){
        replier.reply("Hello! Nice to meet you!");
    }
}
\end{lstlisting}
그러면, 발신자가 오직 'Hello'라고 발신하였을 때만, 챗봇은 'Hello! Nice to meet you!'
라고 답장할 것이다. 이와 같이 조건문을 사용하여 발생할 수 있는 상황에 대해 꾸준히 작성하면
챗봇도 같이 점차 적절한 발화를 할 수 있게 될 것 이다. 이 책에서는 조건문에 대해서 따로 구문을
설명하지는 않으니, 이 책에서 나와있는 예제들을 통해 몸으로 익히도록 하자. 물론, 따로 프로그래밍
관련 서적을 참고하는 것도 매우 좋다. 아니, Best다!

\section{Long Scripts Problem}
조건문을 이용하여 열심히 챗봇을 개선해 나가다 보면 다음과 같은 코드를 만나기 십상이다.
\begin{lstlisting}[language=JavaScript,escapeinside=~~]
function response(room, msg, sender, isGroupChat, replier, ImageDB, packageName, threadId){
    if (msg=='Hello'){
        replier.reply("Hello! Nice to meet you!");
    }
    if (msg=='loli is'){
        replier.reply("LIFE!!!");
    }
    if (msg=='XD'){
        replier.reply("lol");
    }
}
\end{lstlisting}
단지 3가지 경우에 적절히 응답하도록 만든 간단한 챗봇임에도 불구하고 11줄이나 차지하는
너무 더럽다고 해도 과언이 아닌 스크립트가 되어 있다. 하지만, 자바스크립트에서는 파이썬과는 달리
줄에 대해서 관대하기에 다음과 같이 쓸 수도 있다.
\begin{lstlisting}[language=JavaScript,escapeinside=~~]
function response(room, msg, sender, isGroupChat, replier, ImageDB, packageName, threadId){
    if (msg=='Hello'  ){replier.reply("Hello! Nice to meet you!");}
    if (msg=='loli is'){replier.reply("LIFE!!!");}
    if (msg=='XD'     ){replier.reply("lol");}
}
\end{lstlisting}
위와 같이 작성하면 5줄만으로도 작성할 수 있다. 그러나, 컴퓨터에서는 사실상 같은 스크립트일 것이며, 이 스크립트 또한
응답할 상황이 많아지면 많아질 수록 스크립트의 내용은 방대해 질 것 이다. 이를 해결하기 위해서는
어떻게 작성을 하면 좋을까? 이에 대한 답은 객체(혹은 딕셔너리)를 이용하는 것이다.
\begin{lstlisting}[language=JavaScript,escapeinside=~~]
var msgDict = {'Hello': "Hello World!", 'loli is': "LIFE!!!", 'XD': "lol"};
function response(room, msg, sender, isGroupChat, replier, ImageDB, packageName, threadId){
    if (msg in msgDict){
        replier.reply(msgDict[msg]);
    }
}
\end{lstlisting}
물론 독자가 Switch문에 대해서 알고 있다면, 다음과 같이 작성할 수는 있다.
\begin{lstlisting}[language=JavaScript,escapeinside=~~]
function response(room, msg, sender, isGroupChat, replier, ImageDB, packageName, threadId){
    switch(msg){
        case 'Hello' :
            replier.reply("Hello! Nice to meet you!");
            break;
        case 'loli is' :
            replier.reply("LIFE!!!");
            break;
        case 'XD' :
            replier.reply("lol");
            break;
    }
}
\end{lstlisting}
음... 그냥 객체를 이용하자.
객체에 대한 내용은 차후에 다시 언급될 것이고 무엇보다 매우 편리한 것이니 반드시 숙지해두자!
이것으로 단순한 자동응답 챗봇은 구현할 수 있게 되었다.
이제 본격적으로 챗봇을 개발해보자! 다음 챕터부터는 챗봇에 여러 기능을 추가하게 될 것이다.

\chapter{}

\section{Random Replication}

만약에 발신자가 '뭐임'라고 할 때마다 '글쎄'라고 챗봇이 답장을 하게 되면, 답장자가 챗봇이라는
느낌을 강하게 받을 뿐만 아니라 if문에 대해서 알고 있는 발신자 또한 개발자라면, 바로 if문을
이용해 구현했다는 것을 알아낼 것이고 너무 무분별한 답장기능은 때로는 특정 채팅방으로부터 강퇴
사유가 되기도 한다. 따라서 각 상황별 여러가지 답장을 할 수 있도록 구현해야 한다. 이 때 사용되는
것이 바로 Math라는 라이브러리이다. 맞다. 독자가 매우 싫어하는 과목, 그 Math이다. 물론, 수학을
보고 흥분하는 변태일 가능성도 있긴 하다. 어찌됬든, 이상한 소리는 여기까지하고 하던 말을 이어가자면
Math라는 라이브러리에는 당연히 수학에 관한 함수나 여러 기능이 포함되어 있는데 이를 이용하여
챗봇이 랜덤하게 답장을 하도록 구현할 것이다.
우선 $\texttt{java.lang.Math.random()}$라는 함수는 0부터 1사이에서의 랜덤한 수를 제공한다.
이 기능은 공학용 계산기에서의 $\texttt{Ran}\sharp$ 과 동일하다.
이를 이용하여 1차원 배열에서 랜덤한 성분을 뽑아내는 함수를 다음과 같이 구현할 수 있다.
\begin{lstlisting}[language=JavaScript,escapeinside=~~, caption={$\texttt{randomItem(array)}$}]
function randomItem(a) {
    return a[java.lang.Math.floor(java.lang.Math.random() * a.length)];
}
\end{lstlisting}
이제 다음과 같이 작성하자.
\begin{lstlisting}[language=JavaScript,escapeinside=@@]
function randomItem(a) {
    return a[java.lang.Math.floor(java.lang.Math.random() * a.length)];
}
var msgDict = {
    @'안녕하세요'@: [
        @"안녕하세여$\sim$!",@
        @"어서와여$\sim$!",@
        @"안녕$\sim$ 이에여$\sim$!"@
    ],
    @'냥!':@ [
        @"냐아앙$\sim$!",@
        @"냥$\sim$! 냥$\sim$!",@
        @"냐아$\sim$!"@
    ],
    @'박제':@ [
        @"어휴;;;",@
        @"증! 거! 확! 보!"@
    ]
};

function response(room, msg, sender, isGroupChat, replier, ImageDB, packageName, threadId){
    var probability = java.lang.Math.random() * 100;
    if (probability >= 70){
        if (msg in msgDict){
            replier.reply(randomItem(msgDict[msg]));
        }
    }
}
\end{lstlisting}
가나다
\end{document}